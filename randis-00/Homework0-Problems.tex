\documentclass[a4paper]{article}
\usepackage[letterpaper, margin=1in]{geometry} % page format
\usepackage{listings} % this package is for including code
\usepackage{graphicx} % this package is for including figures
\usepackage{amsmath}  % this package is for math and matrices
\usepackage{amsfonts} % this package is for math fonts
\usepackage{tikz} % for drawings
\usepackage{hyperref} % for urls

\title{Homework 0}
\date{9/6/16}
\author{John Randis \\ github name: jrands \\ kaggle name: jrands \\ https://github.com/jrands/Artificial-Intelligence}


\begin{document}
\lstset{language=Python}

\maketitle

\section{Solution to Problem 1}
Derivative: 0 = -6(x-4)
Max: g(x) = 18 at x = 4

\section{Solution to Problem 2}


\begin{equation}
  f(x) with respect to x_{0} = 9x_{0}^2 - 2x_{1}^2
\end{equation}
\begin{equation}
  f(x) with respect to x_{1} = -4x_{0}x_{1} + 4
\end{equation}

\section{Solution to Problem 3}
\subsection {}
The two matricies cannot be multiplied, because the columns of the first matrix exceed the rows of the second. 

\subsection {}
\begin{lstlisting}
A^T=[1, 2]
[4, -1]
[-3, 3]
\end{lstlisting}

Python verification: 

\begin{lstlisting}[frame=single]
import numpy as np

a = [[1, 2], [4, -1], [-3, 3]]
b = [[-2, 0, 5], [0, -1, 4]]

np.dot(a, b)

\end{lstlisting}

\begin{lstlisting}
array([[-2, -2, 13],
       [-8,  1, 16],
       [ 6, -3, -3]])
\end{lstlisting}

\section{Solution to Problem 4}

Simple Gaussian - A function with normal distribution that represents data in a symmetrical bell shaped graph.

Multivariate Gaussian - Mutlivariate Gaussian distribution is the distribution of the Simple Gaussian to higher dimensions. 

Bernoulli - A distribution having two possible outcomes in which success occurs with a certain probability and failure occurs with a certain probability.

Binomial - A distribution having only one outcome for each trial, with each having the same probability of success making the trial mutually exclusive. 

Exponential - Given a Poisson distribution with a rate of change λ, the distribution of waiting times between successive changes (with k = 0) is:
\begin{equation}
D(x) \equiv P (X < x)
	= 1 - P (X > x)
	= 1 - e^-λx
\end{equation}

\section {Solution to Problem 5}

(graduate only)

\section{Solution to Problem 6}

The expected value is 2

\section{Solution to Problem 7}
\subsection{}
If y = 1.1 and Z = N, x* is 2.2
\subsection{}
x* is located at the midpoint between y and Z


\end{document}
